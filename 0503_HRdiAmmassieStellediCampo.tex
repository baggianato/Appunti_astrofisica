\section*{03-05 - Diagrammi HR di Ammassi e Stelle di campo}
Uno dei diagrammi più importanti dell'astrofisica è il diagramma Hertzsprung - Russell (HR). Ha sull'asse delle ordinate una quantità legata alla luminosità, la magnetudine. Sulle oscisse avremo la controparte osservativa della temperatura effettiva: indice di colore o il tipo spettrale. 
\subsection*{Ammassi aperti}
Vediamo (a pag 2/49 HR) delle sequenze diverse. I primi due sono diagrammi di due ammassi (Pleiadi e Iadi), mentre il terzo sono stelle di campo vicine al sole. Dove sta la differenza? I primi due sono oggetti che nascono dalla stessa nube, quindi con composizione chimica molto simile, e ad un tempo simile (le stelle sono coeve). Queste condizioni rendono gli ammassi stellari dei laboratori di astrofisica che permettono la verifica dei modelli. In particolare le Pleiadi appartengono alla categoria degli ammassi aperti. Analizzando il diagramma HR delle Pleiadi notiamo che le stelle si dispongono sulla sequenza principale. Sappiamo quindi che sono stelle che bruciano idrogeno nel core. Ciò che differenzia queste stelle è la massa, che sappiamo essere proporzionale a luminosità e alla temperatura effettiva. Vediamo che a destra della sequenza principale si dispongono delle stelle. Sono binarie non risolte! Quindi la magnitudine che misuriamo, pensiamo che sia quello di una singola stella, ma in realtà è quella di una coppia. Guardando invece il diagramma delle Iadi, osserviamo una sequenza monoparametrica (cambia solo la massa). Vediamo, a differenza che per le Pleiadi, una parte superiore in cui è presente una curva, differenza dovuta all'età dell'ammasso. Esiste anche una differenza in composizione chimica, ma è un effetto secondario rispetto all'età. Guardando la figura a pag. 13/39, come si spiega che l'ammasso NGC 2362 ha una sequenza principale che arriva a luminosità molto più alte rispetto a M67? Abbiamo visto che per le stelle in sequenza principale il tempo di vita decresce al crescere della massa. \\
Per vedere come cambia la posizione di una stella a massa fissata in un diagramma HR teorico, abbiamo quelle che si chiamano tracce evolutive (pag. 15/49). Qual è la controparte teorica di un diagramma HR evolutivo di un ammasso? Non è la traccia evolutiva ma è l'isocrona (pag. 16/49). Non è altro che il luogo dei punti che hanno stessa età, stessa composizione chimica iniziale ma masse diverse. Guardando la figura, ci rendiamo conto che il turn off sia un ottimo indicatore di età. \'E lo strumento teorico che ci permette di datare i sistemi stellari. Uno prende le isocrone teoriche, le posiziona sul diagramma osservativo e si fa un fit.\\
Forma non ben definita, difficilmente superano qualche migliaio di stelle, range ti età molto vasto.
\subsection*{Ammassi globulari}
Il termine globulare viene dalla caretteristica simmetria sferica. Hanno tra alcune decine di migliaia fino a milioni di stelle. Sono tutti vecchi, nella nostra galassia non esistono ammassi globulari giovani (hanno età maggiori di $10^{10}$ anni) e sono per questo fondamentali per conoscere le fasi iniziali della galassia. Ci aspettiamo che la loro composizione chimica sia diversa rispetto agli ammassi più giovani, infatti sono poveri di metalli. Essendo così antichi ci aspettiamo anche che la sequenza principale sia molto più corta (pag. 20/49). C'è voluto molto tempo per riuscire a risolvere le singole stelle dell'ammasso perché mediamente sono molto lontani da noi ($2\um{kpc}$ il più vicino). In sequenza principale avremo stelle con luminosità più bassa e magnitudine apparente più grande. Inoltre nelle immagini c'è il problema dell'affollamento (crowding). Una grossa differenza nei diagrammi HR è che i globulari hanno un vistoso ammasso delle giganti rosse ed un ramo orizzontale che negli aperti non abbiamo visto. Quest'ultimo è molto "orizzontale" e molto esteso perché le stelle hanno, nel ramo di giganti rosse, perché perdono massa in modo stocastico. \\
A differenza degli ammassi aperti, negli ammassi globulari abbiamo popolazioni di stelle con età diversa e diversa composizione chimica.
\subsection*{Stelle di campo}
Guardando un diagramma di stelle di campo notiamo una grande dispersione (pag. 34/49). Continuiamo a vedere le fasi principali ma osserviamo una sequenza principale molto più larga, analogamente nel ramo delle giganti. Il motivo è proprio dettato dal fatto che queste stelle sono il risultato di molte generazioni distinte di stelle, con età e composizioni chimiche molto diverse. La DR2 di GAIA ha rivoluzionato anche questo diagramma (pag. 38/49). Si riescono a parallassare distanze maggiori e vedere stelle a grandi magnitudini. Notiamo che aumenta la larghezza della sequenza principale e che compaiono parti che prima non vedevamo, come la sequenza delle nane bianche.\\
Se non abbiamo l'informazione sulla parallasse vediamo una situazione molto più complicata (pag. 40/49). Abbiamo a che fare con le magnitudini apparenti, quindi una stella ci può sembrare più luminosa di un'altra nonostante non lo sia, semplicemente perché più vicina a noi, ed il diagramma diventa molto più difficile da interpretare. \\

\section*{Galassie nane}
A pag 41/49 vediamo un primo esempio di una galassia nana. È una galassia satellite della via lattea e dista $\sim 140\um{kpc}$ da noi. I puntini che vediamo non sono stelle ma ammassi globulari di questa galassia. Fino a poco tempo fa non riusciva a costruire un diagramma HR di questa galassia poiché la distanza era tale da non riuscire a risolverne le stelle. Col HST siamo riusciti a risolverle. Notiamo alcune differenze (pag. 43/49): non vediamo la sequenza principale, perché è molto lontano. Con un'integrazione maggiore si sarebbe potuta vedere una parte più bassa. E' fondamentale rendersi conto del fatto che l'universo è garnde e non riusciamo a risolvere stelle troppo lontane o non abbastanza luminosa. Tornando al diagramma, notiamo che ci sono più fasi evolutive. Si capisce perché notiamo sia una parte abbastanza giovane di sequenza principale, sia il ramo delle giganti rosse. 

\section*{Binarie}
(Pag. 45/49) Immaginiamo di avere due stelle che orbitano attorno al comune centro di massa. Immaginiamo di non riuscire a risolverle. Riusciamo ad accorgerci che siano una coppia guardando gli spettri e analizzando l'effetto Doppler che, a differenti tempi, farà sì che si abbiano spettri diversi. In questo caso si parla di \textit{Binarie spettroscopiche}. Se il piano dell'orbita è lungo la linea di vista, succede che le due stelle periodicamente passino una d'avanti all'altra. Nonostante non siamo in grado di risolvere questa coppia, osserviamo una variazione della luminosità del puntino. Parliamo di \textit{Binarie ad eclisse} (pag. 46/49). Quindi ci rendiamo conto che questo è un sistema binario con strumenti fotometrici. Le binarie ad eclisse sono molto interessanti poiché sono gli unici sistemi per i quali si riesce a dare una misura precisa e accurata di massa. Se uno riuscisse a misurare anche gli spettri, quindi misura delle velocità radiali, riuscirebbe a misurare anche il raggio. Sono quindi oggetti che riescono a vincolare i modelli teorici. Un'altra tipologia di binarie è quella di \textit{Binarie astrometriche} nelle quali si fanno misure, appunto, astrometriche cioé di posizione nel cielo. A pag. 48/49 vediamo una misura del moto proprio della stella Sirio (spostamento nella direzione ortogonale alla direzione di vista). Questa immagine è ciò che ha permesso a Bessel di capire, un paio di secoli fa, che Sirio non sia una stella singola ma appartiene ad un sistema binario. E vedremo che la compagna di Sirio, Sirio B, è stato il primo esempio di nana bianca.