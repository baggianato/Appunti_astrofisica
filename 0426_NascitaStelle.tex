\section*{26-04 - Nascita di una stella}
In seguito all'instabilità di Jeans, le nubi cominciano a collassare sotto l'effetto della gravità. Succede che il raggio diminuisce e aumenta la densità della nube, inizialmente molto rarefatta, fino a che la materia non diventa fortemente opaca e intrappolando la radiazione al suo interno si scalda. Parte della materia forma un disco di accrescimento (dovuto ad una piccola rotazione che poi viene amplificata). Man mano che il tempo passa il disco viene in parte spazzato via e con quello che resta si formano pianeti e sistema planetario. Abbiamo delle conferme guardando nella regione più profonda della nebulosa di Orione. Nel visibile non riusciamo a penetrarla ma nell'infrarosso notiamo la presenza di stelle molto giovani, immerse ancora nella nube da cui sono nate. Non c'è ancora stato il tempo per spazzarla via. Grazie al HST sono stati scoperti i primi dischi protoplanetari. (pag. 19/83 evoluzione-stellare) \\
Cosa succede alla stella? Una volta spazzato via il gas, la stella si troverà in una condizione di grande espansione e di $T_{\text{eff}}$ relativamente bassa. In un diagramma HR si trova nella regione in alto a destra (alta luminosità e bassa temperatura). Questa zona prende il nome di presequenza principale. (pag. 28/83). Si evolve col teorema del viriale, con tempi scala di Kelvin Helmoltz, contraendosi e aumentando la propria temperatura. Quando arriverà alla temperatura intorno al milione di gradi, verrà bruciato il deuterio ($p+d$) prima di iniziare la $p+p$. Questa fase dura poco e vediamo che nel diagramma HR la traccia verticale subisce una piccola deviazione. In che condizione si trova la stella nella fase verticale? Bassa temperatura effettiva: gran parte degli atomi di idrogeno nell'atmosfera esterna sono neutri o parzialmente ionizzati, quindi abbiamo un inviluppo in una situazione di instabilità convettiva: viene violato il criterio di Schwarzschild (l'opacità è alta, aumenta il gradiente radiativo che supera quello adiabatico e quindi viola il criterio). Quindi nella fase verticale tutta la stella è convettiva e di conseguenza chimicamente omogenea. La traccia verticale si chiama Traccia di Hayashi. Essa è una sorta di confine, nel senso che non esistono strutture all'equilibrio idrostatico per una stella di data massa a destra della traccia. \\
Sappiamo che al crescere della temperatura l'opacità diminuisce (legge di Kramers) e quindi non viene più violato il criterio di Schwarzschild. Si creerà quindi una zona radiativa nel punto più caldo, nel centro. Quando si sviluppa un nucleo radiativo, la stella si allontana dalla fascia di Hayashi e si raggiungono temperature per innescare $p+p$. All'inizio gli elementi secondari (deuterio, elio, ...) non sono all'equilibrio. Quando anche questi raggiungono l'equilibrio la stella si posiziona nel punto di ZAMS (Zero Age Main Sequence). Da quel momento in poi inizia la fase di sequenza principale (combustione centrale dell'idrogeno). Per una stella come il Sole la presequenza dura 30 milioni di anni, mentre la fase di sequenza principale dura circa 10 miliardi di anni. Naturalmente stelle di massa diversa avranno durate diverse. (pag. 29/83). Cambia il rapporto tra zona orizzontale e zona verticale. Più grande è la massa e meno tempo starà sulla fascia di Hayashi, perché sono più calde. \\
Riassumendo: fase vericale struttura convettiva, zona orizzontale core radiativo, poi si posiziona sulla sequenza principale.