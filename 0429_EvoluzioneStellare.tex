\section*{29-04 - Evoluzione stellare}
Illustreremo le principali fasi evolutive della stragrande maggioranza delle stelle. \\
Guardando un'immagine del Sole (nell'UV) notiamo una granularità dovuta ai moti convettivi. Nell'inviluppo (parte esterna) è instabile convettivamente, viene violato il criterio di Schwarzschild. Il motivo è riconducibile all'alta opacità della zona. Nel caso del Sole questo inviluppo si estende per circa il $30\%$ del raggio. Mentre le zone centrali sono all'equilibrio radiativo (core all'equilibrio). Non tutte le stelle sono fatte così. La distinzione principale dipende dalla massa, che risulta il parametro dominante per le caratteristiche principali delle stelle. Conta anche la composizione chimica ma meno. Studiando le reazioni nucleari abbiamo visto che abbiamo due canali: $pp$ e $\ce{CNO}$. Quest'ultima ha bisogno di temperature più alte. La dipendenza della temperatura nella produzione di energia nucleare è molto più spinta nella $\ce{CNO}$: $pp$ va come $T^4$, $\ce{CNO}$ come $T^{15}$. Per masse abbastanza piccole (tra $0.4M_{\astrosun}$ e $1.2M_{\astrosun}$), in sequenza principale non si ha temperatura abbastanza alta per avviare la $\ce{CNO}$ e si ha quindi $pp$ che avviene nel core. Ma nelle stelle abbastanza calde ($T_c\approx 15\times10^6$, e di conseguenza massicce($>1.2M_{\astrosun}$)), la $\ce{CNO}$ diventa poi domninante. Quindi nelle stelle abbastanza piccole, in cui avviene solo $pp$ nel core, quest'ultimo è all'equilibrio radiativo perché il flusso non è sufficientemente intenso da far diventare $\nabla_{rad} > \nabla_{ad}$. Nelle stelle sufficientemente grandi avviene $\ce{CNO}$, e alle temperature alle quali si ha questo tipo di produzione spostandosi dal centro verso l'esterno ci sarà una variazione molto brusca come produzione di energia nucleare, che sarà molto più concentrata nel core. Ciò fa sì che si abbia un flusso molto più intenso che fa crescere il $\nabla_{rad} > \nabla_{ad}$. Da notare come una proprietà microscopica (efficienza di produzione $\ce{CNO}$ rispetto a $pp$) ha un effetto macroscopico. Per quanto riguarda l'inviluppo invece si ha la situazione opposta: le stelle piccole hanno inviluppo convettivo, quelle grandi inviluppo radiativo. Le stelle più piccole hanno l'inviluppo convettivo poiché la zona è più fredda con temperature alle quali c'è una parziale ionizzazione (che implica alta opacità). Le stelle più grandi sono abbastanza calde da avere l'inviluppo completmaente ionizzato. La massa che discrimina questi due comportamenti è circa $1.2M_{\astrosun}$. \\
Scendendo ancora in massa (0.4-0.3 masse solari), succede che la stella diventa completamente convettiva. L'inviluppo convettivo affonda sempre di più fino a coprire l'intera struttura. (pag38/83 evoluzione-stelle). \\
Sappiamo che i moti convettivi avvengono con tempi scala molto brevi rispetto al tempo scala nucleare. Quindi ci aspettiamo che le regioni convettive siano chimicamente omogenee (pag 40/83 ma spiegano bene la cosa i grafici a pag. 4 e 47.). Come mai ci fermiamo ad una massa di $(0.09\div0.08)M_{\astrosun}$? Perché la stella, se non ha sorgenti nucleari attive, si contrae in risposta alla perdita di energia per luminosità, quindi aumenta la temperatura centrale e aumenta la densità. Può succedere che prima che la stella raggiunga le temperature necessarie per la combustione dell'idrogeno, la stella raggiunge densità sufficienti per la configurazione di degenerazione elettronica. Per masse minori di $0.08\sunmass$ la stella degenera prima di iniziare a bruciare l'idrogeno. Al posto di una stella si viene a creare una Brown Dwarf (Nana bruna), che è una via di mezzo tra una stella ed un pianeta.\\
Sappiamo che una relazione importante è quella massa luminosità $L\propto M^3$. A pag 42 abbiamo i risultati esatti, ed effettivamente in certi range di masse si ha un andamento a legge di potenza con una potenza che varia intorno a 3.\\
Una volta che la stella è in sequenza principale brucia l'idrogeno con i secondari all'equilibrio. Man mano che passa il tempo finirà l'idrogeno nel centro e di conseguenza anche la fase principale. A seconda della massa della stella, si avrà un'evoluzione diversa. (pag. 44/83). La stella si sposta a temperature più alte (sx) e poi aumenta la sua luminosità (alto). A quel punto la traccia curva verso destra, diminuendo quindi la temperatura. Il punto in cui ciò avviene prende il nome di \textit{Turn Off}. In quel punto si ha la temperatura effettiva più alta e corrisponde al momento in cui l'idrogeno finisce. Questa curva è molto dolce, dato che l'idrogeno finisce prima al centro e si deve spostare ad una shell. Il core è quindi diventato di elio, circondato da una shell di idrogeno che brucia. Le stelle che bruciano con $\ce{CNO}$ non hanno lo stesso andamento, ma hanno un andamento a zig zag che prende il nome di \textit{overall contaction} (pag 46/83) ed è dovuta al fatto che la combustione avviene in una zona convettiva. Infatti l'idrogeno non viene bruciato solo al centro, ma in tutta la zona in cui c'è convezione. Questa zona è più grande della zona in cui ci sono le temperature necessarie per bruciare l'idrogeno, quindi quando quest'ultimo termina e si crea il core di elio, nella shell successiva la temperatura non è abbastanza alta per innescare la combustione dell'idrogeno (pag. 47/83). Di conseguenza non si ha il passaggio dolce dal core alla shell, quindi la stella contrae con tempi scala di KH e si ha l'andamento a zig zag. Vediamo come varia il tempo di vita in funzione della massa nel grafico a pag 58/83. \\
Il core della stella appena uscita dalla sequenza principale è formato da elio. Per innescare la $3\alpha$ ci vogliono centinaia di milioni di gradi, quindi in questa fase il core è inerte poiché non ci sono le temperature adeguate perché la reazione avvenga. Attorno al core abbiamo una shell in cui brucia l'idrogeno con temperature abbastanza elevate perché avvenga il $\ce{CNO}$. Attorno alla shell abbiamo l'inviluppo. Quando si accende la shell di idrogeno, l'inviluppo si espande moltissimo aumentando il raggio della stella di qualche ordine di grandezza (nel caso del sole di circa 100 volte). La stella diventa quindi una gigante rossa, il cui core è molto piccolo in raggio, molto molto denso, mentre si ha un enorme inviluppo estremamente rarefatto. Quando il sole diventerà una gigante rossa, nel core di elio si avranno densità dell'ordine del milione di grammi per centimetro cubo mentre la densità in una zona media dell'inviluppo sarà minore di quella dell'aria che respiriamo. La shell di idrogeno produce elio che nel tempo fa aumentare la massa del core. Quindi nella fase di gigante rossa, la combustione nucleare è solo in shell e questa combustione determina un progressivo aumento della massa del core di elio (struttura a pag. 51/83). L'inviluppo sarà convettivo. Cosa fa nel diagramma HR? Uscita dalla sequenza principale, passa dal turn off e va verso destra \textit{(fase di sub gigante}) e poi sale a luminosità progressivamente più alte (\textit{ramo delle giganti rosse}). Nel ramo delle giganti rosse il core si sta riscaldando, fino al punto in cui non si arriva alle temperature che inneschino la combustione centrale dell'elio, la $3\alpha$. Questo innesco può avvenire in due modi distinti a seconda della massa della stella. Per masse non troppo grandi ($<2.3\sunmass$) il core di elio diventa degenere in gigante rossa. Per stelle di massa più grande il core non è degenere. Noi sappiamo che un mezzo degenere si comporta in maniera diversa rispetto ad uno non degenere in risposta all'innesco di una reazione nucleare. Prendiamo una situazione non degenere: accendendo le reazioni termonucleari si libera energia, aumenta la temperatura e, siccome il mezzo non è degenere, aumenta la pressione e la regione risponde espandendosi e quindi diminuisce la temperatura. C'è un effetto che tende a calmare l'innesco e si raggiunge una situazione di equilibrio. Viceversa, in un mezzo degenere sappiamo che la pressione è largamente dominata dalla pressione degli elettroni degeneri, ma in quel caso la dipendenza dalla temperatura è praticamente assente. Quindi nel momento in cui inneschiamo la reazione termonucleare liberiamo energia, aumenta la temperatura ma, dato che la pressione non dipende dalla temperatura per un mezzo degenere, la regione non si espande e tutta l'energia liberata fa aumentare la temperatura, aumenta quindi il rate delle reazioni nucleari e abbiamo un meccanismo che si autoincentiva. Questo meccanismo si chiama \textit{flash dell'elio} ed è ciò che succede in stelle che hanno una massa inferiore a $2.3\sunmass$. Il vertice del ramo delle giganti rosse è in corrispondenza del flash dell'elio. Viene liberata un'enorme energia che non porta alla distruzione della stella ma alla rimozione della degenerazione. Nelle stelle di massa più grande invece, il core non è degenere, l'innesco avviene in maniera più controllata ed a luminosità più basse perché il core, non essendo degenere, si scalda più facilmente. In entrambi i casi, come risultato si ha che la stella passa in una fase evolutiva in cui si avvia la combustione dell'elio nel core. Abbiamo quindi due sorgenti di energia nucleare: il core di elio e la shell di idrogeno. La combustione dell'elio avviene in un core convettivo, poiché la dipendenza dalla temperatura della $3\alpha$ è molto grande ($\sim T^{40}$). (Approfondire cosa succede nel diagramma HR durante la combustione dell'elio). Finita la combustione dell'elio si ha la fase di \textit{Ramo Asintotico Gigante (AGB)} in cui si ha il passaggio dalla combustione in core alla combustione in shell dell'elio, attorno ad un core inerte costituito da $\ce{C}$ e $\ce{O}$  (pag. 60/83). Il Sole passerà per quetsa fase evolutiva, alla fine della quale non riuscirà ad innescare la combustione del core C-O. Man mano che la combustione in fase AGB (Asymptotic Giant Branch) procede, il core C-O aumenta in massa e ci troviamo di fronte ad una biforcazione: riuscirà la stella ad innescare C-O? Dipenderà dalla massa ($8\sunmass$ è la frontiera). Ad esempio il sole non ce la farà. Ciò che succede nel caso non si riesca è (pag. 62/83) la stella sale nel HR, non innesca C, l'inviluppo sarà molto molto rarefatto e la stella perderà molta massa per via dei venti solari fino a che, non riuscendo a innescare il C, comincerà a contrarsi e aumenterà la temperatura fino a diventare una \textit{Nana Bianca}. Quest'ultima si evolve nel diagramma HR lungo una traccia diagonale che implica il raggio costante della stella. Per capirlo basta considerare la relazione 
\begin{equation*}
    L = 4\pi R^2\sigma T_{\text{eff}}^4
\end{equation*}
Quindi vediamo come la luminosità sia legata a raggio e temperatura. Avendo nel nostro piano $\log L$ e $\log T_{\text{eff}}$ abbiamo
\begin{equation*}
    \log L = \log(4\pi R^2\sigma T_{\text{eff}}^4)
\end{equation*}
e abbiamo quindi una retta che è a raggio costante. \\
Nel passaggio dalla zona fredda a quella calda la radiazione elettromagnetica emessa dalla stella diventa tale da ionizzare il gas costituito dai venti stellari emessi precedentemente. Questa fase, molto bella, prende il nome di \textit{nebulosa planetaria}. (pag. 64/83) La materia diffusa che osserviamo è l'inviluppo eroso dai venti solari precedentemente che poi viene riscaldata e di conseguenza ionizzata. Si chiama così per motivi storici ma non hanno nulla a che vedere con i pianeti.\\
Le stelle di grande massa ($>8\sunmass$ e quindi che innescano C), se hanno massa >10 masse solari riescono ad innescare tutte le reazioni nucleari successive. Riusciranno quindi ad attraversare tutte le fasi di combustione fino ad arrivare alla combustione del silicio per produrre un nucleo di ferro inerte. (pag. 74/83) Si verrà a formare una struttura a cipolla, dovuta al fatto che l'estensione della regione in cui brucia un elemento sarà più piccola di quella in cui è bruciato l'elemento precedente. Arrivati a quel punto la stella non produrrà più energia con reazioni di fusione nucleare e quello che succederà è che, quando il core di elio diventa abbastanza grande, la stella diventa instabile (instabilità dovuta alla forte degenerazione elettronica del core di ferro, si ha densità così elevata da avere un'energia di fermi di questi elettroni così elevata da innescare i processi beta inversi - lo vedremo -, e sottraendo gli elettroni la pressione diminuisce e quindi inizia un collasso molto rapido (frazione di secondo)). Si arriverà ad una stella di neutroni o buco nero per il nucleo ed il resto verrà eiettato con un'esplosione di supernova (pag. 75/83).